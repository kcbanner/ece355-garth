\documentclass{report}
\usepackage{graphicx}
\usepackage{float}
\usepackage{fullpage}
\usepackage{array}
%\usepackage{pdflscape}
\usepackage{lscape}
%\usepackage[parfill]{parskip}

\setlength{\extrarowheight}{4pt}

\graphicspath{{./images/}}

\floatstyle{boxed}
\restylefloat{figure}

\begin{document}
\begin{titlepage}
\begin{center}
\vfill
\hfill
\\[2cm]
\textsc{\LARGE University Of Waterloo}
\\[1cm]
\textsc{\LARGE ECE 355}
\\[2cm]

\hrule
\hfill
\\[0.5cm]
\textsc{\huge Software Requirements Specification}
\\[0.5cm]
\textsc{\huge GARTH}
\\[0.5cm]
\textsc{\huge Green, Aware, and Responsive Total Home}
\\[0.5cm]
\hrule
\hfill
\\[1cm]
\textsc{\LARGE Group 16} \\[0.4cm]

\begin{minipage}{0.4\textwidth}
\begin{flushleft} \large
Ben Ridder \\
Casey Banner \\
Zack MacLennan
\end{flushleft}
\end{minipage}
\begin{minipage}{0.4\textwidth}
\begin{flushright} \large
20302041 \\
20299452 \\
20305946 
\end{flushright}
\end{minipage}


\vfill

{\large \today}
\end{center}
\end{titlepage}

\tableofcontents
\listoffigures

\chapter{Introduction} % 10% Zack
\label{ch:introduction}

\section{Executive Summary}

\section{Purpose}

\section{Scope}

\section{Assumptions}

\section{Changes to Requirements}

\section{Design Goals}

\section{Prioritization of Functionality}

\section{Terminology and Definitions}
\begin{description}
\item[RAID]
\item[Parity disk]
\item[rsync]
\end{description}

%\section{References}

\chapter{Architecture} % 20%
\label{ch:architecture}

\section{Overview}

\section{Subsystem Decomposition}

\chapter{System Design} % 10% Zack
\label{ch:system-design}

\section{Hardware/Software Mapping}

\section{Data Resource Management}
%TODO mentioned proprietary software that should be fleshed out in this SDD

The two main components to GARTH's data resource management are the local storage
hardware and the remote server. The local storage component is contained within the
home and consists of a Linux server that will be stored in a secure location. The Linux server
is a desktop computer with four 2TB hard drives arranged in a modified RAID array with a
parity disk for redundancy. In this case, if a hard drive should fail the system will continue
to run as expected until a replacement hard drive gets installed. The server will use proprietary 
software  to create TXT files that log system events as well as store video that is captured by the 
cameras situated within and outside the home.

The remote server is back-up storage that is located offsite at GARTH headquarters. It is
merely a remote backup of system logs and important video data that should be saved for
future reference. This includes any log-files or video that was recorded by the system during a 
critical security violation. This synchronization will be accomplished by using \textbf{rsync} 
network protocol installed on both the local and remote server. The data that is backed up
remotely will be stored in a SQL database such that storing and querying important data
can be accomplished easily.

\section{Access Control and Security}

\section{Global Software Control}

\section{Boundary Conditions}

\chapter{Interfaces} % 10%
\label{ch:interfaces}

\section{External System Interfaces}

\section{Internal Subsystem Interfaces}

\chapter{Object Design} % 30%
\label{ch:object-design}

\section{Design Patterns}

\section{Algorithms}

\section{Packages}

\section{Object and Interface Design}

\section{Dynamic Design Model}

\chapter{Design Evaluation} % 10% Zack
\label{ch:design-evaluation}

\section{Design Trade-offs}

\section{Re-use}

\section{Optimizations}

\section{Extensibility}

\chapter{Operating Environment} % 10%
\label{ch:operating-environment}

\section{Development Platform}

\section{Runtime Platform}

\section{Process Model}

\section{Synchronization}

\section{Fault Handling}

\end{document}
